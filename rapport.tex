\documentclass[french]{article}
\usepackage[utf8]{inputenc}
\usepackage[T1]{fontenc}
\usepackage{babel}
\usepackage{amsmath}
\usepackage{graphicx}
\usepackage{amssymb}
\usepackage{mathabx}
\usepackage{minted}
\usepackage{framed}
\usepackage{fancyhdr}
\renewcommand{\MintedPygmentize}{/Library/Frameworks/Python.framework/Versions/3.8/bin/pygmentize}
\usepackage{geometry}
\geometry{hmargin=2cm,vmargin=3cm}
\pagestyle{fancy}
\lhead{Benoît GUILLEMET}
\rhead{}

\begin{document}

\part*{\begin{center} Exercice\end{center}}

\section{Implémentation}

\begin{framed}
\begin{minted}{ocaml}
type 'a my_list =
| Nil
| Cons of 'a * 'a my_list

let string_of_list str_fun l =
        let rec string_content l = match l with
                |Nil -> ""
                |Cons (a, Nil) -> str_fun a
                |Cons (a, q) -> (str_fun a) ^ ", " ^ (string_content q)
        in "[" ^ (string_content l) ^ "]"

let rec hd l = match l with
        |Nil -> failwith "liste vide"
        |Cons (a, q) -> a

let rec tl l = match l with
        |Nil -> failwith "liste vide"
        |Cons (a, q) -> q

let rec length l = match l with
        |Nil -> 0
        |Cons (a, q) -> 1 + (length q)

let rec map f l = match l with
        |Nil -> Nil
        |Cons (a, q) -> let reste = (map f q) in Cons ((f a), reste)
\end{minted}
\end{framed}

\section{Exemples}

Le type \texttt{my\_list} est une implémentation alternative des listes standard de OCaml. On définit la liste contenant 1, 3, et 42 de la manière suivante en faisant appel au constructeur \texttt{Cons}.

\begin{leftbar}
\begin{minted}{ocaml}
open My_list
let l = Cons (1, Cons (3, Cons (42, Nil)));;
\end{minted}
\end{leftbar}

La fonction \texttt{string\_of\_list} crée une chaîne de caractères à partir d'une \texttt{'a my\_list}. Elle prend en paramètre une fonction \texttt{str\_fun} qui doit renvoyer une chaîne de caractères à partir d'un élément de type \texttt{'a}.

\begin{leftbar}
\begin{minted}{ocaml}
let string_l = string_of_list string_of_int l in
print_string string_l;;
(* affiche "[1, 3, 42]" *)
\end{minted}
\end{leftbar}

La fonction \texttt{hd} renvoie le premier élément de la liste placée en paramètre, tandis que la fontion \texttt{tl} renvoie la queue de la liste (c'est-à-dire la liste privée de son premier élément). Si la liste est vide, elle lève une exception \texttt{Failure "liste vide"}.

\begin{leftbar}
\begin{minted}{ocaml}
let a = hd l in
print_int a;;
(* affiche "1" *)
let q = tl l in
print_string (string_of_list string_of_int q);;
(* affiche "[3, 42]" *)
\end{minted}
\end{leftbar}

La fonction \texttt{length} renvoie la longueur de la liste placée en paramètre.

\begin{leftbar}
\begin{minted}{ocaml}
let n = length l in
print_int n;;
(* affiche "3" *)
\end{minted}
\end{leftbar}

Enfin, la fonction \texttt{map} prend en argument une fonction \texttt{f : 'a -> 'b} et une \texttt{'a my\_list}, et renvoie la \texttt{'b my\_list} qui contient les images des éléments de la liste par \texttt{f}.

\begin{leftbar}
\begin{minted}{ocaml}
let l2 = map (fun x -> 2*x) l in
print_string (string_of_list string_of_int l2);;
(* affiche "[2, 6, 84]" *)
\end{minted}
\end{leftbar}

\end{document}